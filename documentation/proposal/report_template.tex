%%
%% Edit the following two statements with the title of your report and the names of your team members.

\def\mytitle{The Star Solution}

\def\myauthor{Nick Mako, Dalton Burke, James Vance, \\
and Komi Agbo}


%%=========================================
%%
%%  Do no edit the following block
%%
%%VVVVVVVVVVVVVVVVVVVVVVVVVVVVVVVVVVVVVV

\ifdefined\CLINICMAIN
(by \myauthor)

\else

\documentclass{article}
\usepackage{graphicx}
%\input{clinicPreamble}
\graphicspath{{images/}}
%
\title{\mytitle}
\author{\myauthor}
\begin{document}
\maketitle
\fi
%
% ^^^^^^^^^^^^^^^^^^^^^^^^^^^^^^^^^^^^^^^^
%  DO NOT EDIT ABOVE THIS LINE
%===========================================

%  Your Report Starts Here.
\graphicspath{{images/}}

\section*{Abstract} Your abstract should concisely state what is in your report.
\\
\newline
\noindent \textbf{\textit{Keywords - key1, key2, key3}}

\newpage

\section{Introduction} \label{grp1:intro}

Sam's Hauling, Inc. provides small dumpsters of various sizes to homeowners, contractors, realtors and property managers throughout the metro Denver area. They have a limited number of trucks with which to do these pickups and deliveries and in addition some customers set time windows for said pickups and deliveries. In addition, they have multiple depots that they store the dumpsters and the trucks have varying capacities. Currently, the pickup and delivery of these dumpsters is scheduled by a single individual using only Microsoft Excel. This method for solving it is very time intensive and is less likely to be the most efficient solution. It is quite likely then, that the implementation of some heuristic or metaheuristic could increase the efficiency of this process by a significant margin.

At first glance this problem seems like some variation of the Vehicle Routing Problem. Our goal is to simplify the problem in a manner that would make it more like a Traveling Salesman Problem and thus simpler to solve for an optimal solution. In order to do this, we will first implement what we are calling the Star Method. Following the successful implementation of the Star Method given the necessary constraints of the problem we will then use what we call the Triangle Method


\section{Background} \label{grp1:background}

Our solution was based most closely on the Travelling Salesman Prob
\section{Methods} \label{grp1:methods}

Describe what you did.

\section{Results} \label{grp1:results}

Report on your results. 

\includegraphics[scale=.5]{gplot.png}

\section{Discussion} \label{grp1:discussion}

Discuss your results

\section{Conclusions/Future Work}
Summarize and conclude and describe any future work that you might do.


 \label{grp1:conclusions}

%%=======================================
%%  DO NOT EDIT BELOW THIS LINE
%%
%% VVVVVVVVVVVVVVVVVVVVVVVVVVVVVVVVVVVVVV
\ifdefined\CLINICMAIN
\else
\end{document}
\fi
% ^^^^^^^^^^^^^^^^^^^^^^^^^^^^^^^^^^^^^^^^
%  DO NOT EDIT ABOVE THIS LINE
%=========================================

