%%
%% Edit the following two statements with the title of your report and the names of your team members.

\def\mytitle{The Star Solution}

\def\myauthor{Komi Agbo, Dalton Burke, Nick Mako,\\
and James Vance}


%%=========================================
%%
%%  Do no edit the following block
%%
%%VVVVVVVVVVVVVVVVVVVVVVVVVVVVVVVVVVVVVV

\ifdefined\CLINICMAIN
(by \myauthor)

\else

\documentclass{article}
\usepackage{graphicx}
%\input{clinicPreamble}
\graphicspath{{images/}}
%
\title{\mytitle}
\author{\myauthor}
\begin{document}
\maketitle
\fi
%
% ^^^^^^^^^^^^^^^^^^^^^^^^^^^^^^^^^^^^^^^^
%  DO NOT EDIT ABOVE THIS LINE
%===========================================

%  Your Report Starts Here.
\tikzset{vx/.style={inner sep=1.7pt, outer sep=0pt,circle, fill}}
\tikzset{edge/.style={color=black,line width=.5}}
\graphicspath{{images/}}

\section*{Abstract} Your abstract should concisely state what is in your report.
\\
\newline
\noindent \textbf{\textit{Keywords - key1, key2, key3}}

\newpage

\section{Introduction} \label{grp1:intro}

Sam's Hauling, Inc. provides small dumpsters of various sizes to homeowners, contractors, realtors and property managers throughout the metro Denver area. They have a limited number of trucks with which to do these pickups and deliveries and in addition some customers set time windows for said pickups and deliveries. In addition, they have multiple depots that they store the dumpsters and the trucks have varying capacities. Currently, the pickup and delivery of these dumpsters is scheduled by a single individual using only Microsoft Excel. This method for solving it is very time intensive and is less likely to be the most efficient solution. It is quite likely then, that the implementation of some heuristic or metaheuristic could increase the efficiency of this process by a significant margin.

At first glance this problem seems like some variation of the Vehicle Routing Problem (VRP). It is possible however, to simplify the problem in a manner that would make it more like a Traveling Salesman Problem (TSP) and thus simpler to solve for an optimal solution. In order to do this, we will first implement what we are calling the Star Method. Following the successful implementation of the Star Method given the necessary constraints of the problem we will then use what we call the Triangle Method. After a successful implementation of the Triangle Method with the necessary constraints the next step is to assign routes to each driver.


\section{Background} \label{grp1:background}

Our solution was based most closely on the TSP

\section{Methods} \label{grp1:methods}

Using the way that the landfills and storage locations are visited allows the elimination of the ordering constraint that would be imposed by the VRP. Then treating each service site as a switch allows for the further simplification of the problem. With every service site a switch then drivers would move from the nearest landfill for delivery portion and then back to the nearest landfill after the pickup portion. Filling the routes with this data then creates several \emph{stars} with each landfill being a center and the service sites around it being the points as can be seen in figure 1. Each service site around a star can then be visited in any order without an impact on the cost. 
\begin{center}
\begin{tikzpicture}
  \foreach \y in {0,2} {
    \foreach \x in {0,2} {
      \draw (\x,\y) node[vx]{};
    }
    \draw[edge] (0,0) -- ++(280:.5) node[vx, inner sep=1.2pt]{};
    \draw[edge] (0,0) -- ++(100:.7) node[vx, inner sep=1.2pt]{};
    \draw[edge] (0,0) -- ++(190:.9) node[vx, inner sep=1.2pt]{};
    \draw[edge] (0,0) -- ++(00:.2) node[vx, inner sep=1.2pt]{};
    \draw[edge] (0,0) -- ++(60:.3) node[vx, inner sep=1.2pt]{};
    \draw[edge] (0,0) -- ++(120:.5) node[vx, inner sep=1.2pt]{};

    \draw[edge] (0,2) -- ++(280:.5) node[vx, inner sep=1.2pt]{};
    \draw[edge] (0,2) -- ++(100:.7) node[vx, inner sep=1.2pt]{};
    \draw[edge] (0,2) -- ++(0:.9) node[vx, inner sep=1.2pt]{};
    \draw[edge] (0,2) -- ++(50:.3) node[vx, inner sep=1.2pt]{};
    \draw[edge] (0,2) -- ++(160:.5) node[vx, inner sep=1.2pt]{};

    \draw[edge] (2,2) -- ++(280:.5) node[vx, inner sep=1.2pt]{};
    \draw[edge] (2,2) -- ++(100:.7) node[vx, inner sep=1.2pt]{};
    \draw[edge] (2,2) -- ++(210:.9) node[vx, inner sep=1.2pt]{};
    \draw[edge] (2,2) -- ++(50:.8) node[vx, inner sep=1.2pt]{};
    \draw[edge] (2,2) -- ++(160:.3) node[vx, inner sep=1.2pt]{};
    \draw[edge] (2,2) -- ++(60:.6) node[vx, inner sep=1.2pt]{};

    \draw[edge] (2,0) -- ++(280:.9) node[vx, inner sep=1.2pt]{};
    \draw[edge] (2,0) -- ++(100:.7) node[vx, inner sep=1.2pt]{};
    \draw[edge] (2,0) -- ++(240:.4) node[vx, inner sep=1.2pt]{};
    \draw[edge] (2,0) -- ++(80:.8) node[vx, inner sep=1.2pt]{};
    \draw[edge] (2,0) -- ++(160:.2) node[vx, inner sep=1.2pt]{};
    \draw[edge] (2,0) -- ++(20:.8) node[vx, inner sep=1.2pt]{};
  }
\end{tikzpicture}\\
Figure 1. Stars
\end{center}

Since the problem doesn't actually consist solely of switches it is necessary to expand the star solution to include pickup and dropoffs. This provides a route of landfill, drop-off, pick-up, landfill for a driver, creating a \emph{triangle} (figure 2). These triangle routes allow for further optimization as long as the optimal triangles are chosen.

\begin{center}
\begin{tikzpicture}

  \draw (0,0) node[vx]{};
    \draw[edge] (0,0) -- ++(280:.5) node[vx, inner sep=1.2pt]{};
    \draw[edge] (0,0) -- ++(100:.7) node[vx, inner sep=1.2pt]{};
    \draw[edge] (0,0) -- ++(210:.9) node[vx, inner sep=1.2pt]{};
    \draw[edge] (0,0) -- ++(50:.8) node[vx, inner sep=1.2pt]{};
    \draw[edge] (0,0) -- ++(160:.3) node[vx, inner sep=1.2pt]{};
    \draw[edge] (0,0) -- ++(60:.6) node[vx, inner sep=1.2pt]{};

    \draw[edge] (280:.5) -- (210:.9);
    \draw[edge] (60:.6) -- (100:.7);
\end{tikzpicture}\\
Figure 2. Triangles
\end{center}

To make this decision, we can construct a bipartite graph, one set containing the drop-offs, and the other containing pick-ups, and the existence of an edge between a drop-off and pick-up means that they are \emph{compatible} (their constraints can work together). The weight of the edge between a drop-off and a pick-up is calculated as the difference between doing each normally (star method) and doing them combined as a triangle. In this graph, a matching pairs up drop-offs and pick-ups to be completed as a triangle. A minimum matching in this graph gives us the best possible choice of triangles in order to make the route take the least amount of time.

Due to the nature of the problem simply choosing the most optimum triangle alone isn't enough to give an optimal solution. Transistions between the landfill sites must also be considered. In order to determine each transitio\emph{n min(〖RN〗_i-〖RT〗_i)} for i=0 to i=n is caluculated, where RN is the route cost normally, RT is the cost of doing the transition, and n is the number of routes that are unassigned. In the case where pairs are constrained min(〖RN〗_i-〖RT〗_i) will always be positive since doing the route normally will always be the same or shorter. However in the case where pairs are non-constrained and can be made between zones the cost of doing a route normally can be less than the cost of doing a transition.


\section{Results} \label{grp1:results}

Report on your results. 

\includegraphics[scale=.5]{gplot.png}

\section{Discussion} \label{grp1:discussion}

Discuss your results

\section{Conclusions/Future Work}
Summarize and conclude and describe any future work that you might do.


 \label{grp1:conclusions}

%%=======================================
%%  DO NOT EDIT BELOW THIS LINE
%%
%% VVVVVVVVVVVVVVVVVVVVVVVVVVVVVVVVVVVVVV
\ifdefined\CLINICMAIN
\else
\end{document}
\fi
% ^^^^^^^^^^^^^^^^^^^^^^^^^^^^^^^^^^^^^^^^
%  DO NOT EDIT ABOVE THIS LINE
%=========================================

