%%%%%%%%%%%%%%%%%%%%%%%%%%%%%%%%%%%%%%%%%%%%%%%%%%%%%%%%%%%%%%%%%%%%%%%
% file: clinicmacros.tex  input to clinicF00.tex
% created 12-20-00 by Harvey J. Greenberg
% Last update: 1-3-01
%%%%%%%%%%%%%%%%% Begin new theorems and environments %%%%%%%%%%%%%%%%%
\newcommand{\figref}[1]{Figure~\ref{fig:#1}}
\newcommand{\tabref}[1]{Table~\ref{tab:#1}}
%\newcommand{\eqref}[1]{(\ref{eq:#1})}
\newcommand{\eqrefs}[2]{(\ref{eq:#1})--(\ref{eq:#2})}
\newcommand{\thmref}[1]{Theorem~\ref{thm:#1}}
\newcommand{\lemref}[1]{Lemma~\ref{lem:#1}}
\newcommand{\defref}[1]{Definition~\ref{def:#1}}
\newcommand{\corref}[1]{Corollary~\ref{cor:#1}}
\newcommand{\propref}[1]{Proposition~\ref{prop:#1}}
\newcommand{\asref}[1]{Assumption~\ref{ass:#1}}
\newcommand{\asrefs}[2]{Assumptions~\ref{ass:#1}--\ref{ass:#2}}
\newcommand{\algref}[1]{Algorithm~\ref{alg:#1}}
\newcommand{\probref}[1]{(\ref{prob:#1})}
\newcommand{\secref}[1]{Section~\ref{sec:#1}}
\newcommand{\chapref}[1]{Chapter~\ref{chap:#1}}
\newcommand{\appref}[1]{Appendix~\ref{app:#1}}
\newcommand{\mylabel}[1]{\label{#1}}



\newtheorem{theorem}{Theorem}[section]
\newtheorem{corollary}{Corollary}[theorem]
\newtheorem{lemma}{Lemma}[section]
\newtheorem{proposition}{Proposition}[section]
\newtheorem{conjecture}{Conjecture}[section]
% definition, example and remark environments are defined in 
% clinicPreamble to have Roman text (put term being defined into italics)

\newenvironment{Srule}{\refstepcounter{rule} % cannot use rule for env name
         \medskip\par\noindent\textbf{Rule \therule.} } {\medskip}

\newenvironment{enum0}{\vspace{-.19in}\par
                  \begin{enumerate}\setlength{\itemsep}{-.06in} }
                      {\end{enumerate}}

\newenvironment{exercise}{ \refstepcounter{exercise} 
    \renewcommand{\labelenumi}{\theexercise.\theenumi} 
                          \begin{description}
                            \item [Exercise \theexercise.] }
                         {\end{description} 
    \renewcommand{\labelenumi}{\theenumi} }

% The following needs the following defined:
%    \title, \teacher-name, \teacher-e-mail, \teacher-web, \date
%
\newcommand{\makeclinictitle}{\vspace*{.6in} \begin{center}%
                {\Large\textbf{FINAL REPORT OF THE \\
                               UCDHSC MATHEMATICS CLINIC\\ \bigskip
                               \title}} \\ \bigskip
	                       {\large Taught by \\ \teachername        
                            \\  University of Colorado at Denver
			    \\  and Health Sciences Center 
                            \\ Department of Mathematical Sciences
                            \\ P.O. Box 173364
                            \\ Denver, CO 80217-3364
                            \\ \teacheremail
                     \\ \bigskip \bigskip
                       {\large Sponsored by \\ \sponsorname}
                     \\ \bigskip \bigskip \renewcommand{\arraystretch}{1.1}
                         \begin{tabular}{|c|} \hline
                             \textbf{Participating students:} \\
                              \students \\ \hline
                         \end{tabular} \renewcommand{\arraystretch}{1}
                     \\ \bigskip \bigskip
                              \date} \end{center} \bigskip }
\newcommand{\qed}{\hspace*{\fill}\rule{1.5ex}{1.5ex} }
%%%%%%%%%%%% End newenvironments that might be disabled %%%%%%%%%%%%%%

\newtheorem{algorithm}{Algorithm}
\newenvironment{proof}{ \noindent\textbf{Proof:} }
%                       {\hspace*{\fill}\rule{1.5ex}{1.5ex}  }
                      {                          }
%%%%%%%%%%%% End all newtheorem and newenvironment specs %%%%%%%%%%%%%

%%%%%%%%%%%%%%%%%%%%%%%%%%% Begin defs %%%%%%%%%%%%%%%%%%%%%%%%%%%%%%%

% Simply abbreviations:
%\def\mc{\multicolumn}
%\def\ol{\overline}    % must be in math mode
%\def\ul{\underline}   % "
%\def\ob{\overbrace}   % "
%\def\ub{\underbrace}  % "
%\def\dag{\mbox{$^\dagger$}}
%\def\Larrow{\mbox{$\longleftarrow$}}
%\def\Rarrow{\mbox{$\longrightarrow$}}
%\def\Small{\scriptsize}
%\def\st{\,:}
\def\st{\mbox{subject to}}
% Calligraphic font:
\def\cA{\ensuremath{{\cal{A}}}}
\def\cB{\ensuremath{{\cal{B}}}}
\def\cC{\ensuremath{{\cal{C}}}}
\def\cD{\ensuremath{{\cal{D}}}}
\def\cE{\ensuremath{{\cal{E}}}}
\def\cF{\ensuremath{{\cal{F}}}}
\def\cG{\ensuremath{{\cal{G}}}}
\def\cH{\ensuremath{{\cal{H}}}}
\def\cI{\ensuremath{{\cal{I}}}}
\def\cJ{\ensuremath{{\cal{J}}}}
\def\cK{\ensuremath{{\cal{K}}}}
\def\cL{\ensuremath{{\cal{L}}}}
\def\cM{\ensuremath{{\cal{M}}}}
\def\cN{\ensuremath{{\cal{N}}}}
\def\cO{\ensuremath{{\cal{O}}}}
\def\cP{\ensuremath{{\cal{P}}}}
\def\cQ{\ensuremath{{\cal{Q}}}}
\def\cR{\ensuremath{{\cal{R}}}}
\def\cS{\ensuremath{{\cal{S}}}}
\def\cT{\ensuremath{{\cal{T}}}}
\def\cU{\ensuremath{{\cal{U}}}}
\def\cV{\ensuremath{{\cal{V}}}}
\def\cW{\ensuremath{{\cal{W}}}}
\def\cX{\ensuremath{{\cal{X}}}}
\def\cY{\ensuremath{{\cal{Y}}}}
\def\cZ{\ensuremath{{\cal{Z}}}}

% The following were in latex but are not in latex2e:
\def\Box{\mbox{\begin{picture}(0,0) \put( 2,0){\framebox(7,7)} 
               \end{picture} }}
\def\lhd{\mbox{\large$\triangleleft$}}
\def\rhd{\mbox{\large$\triangleright$}}
\def\unlhd{\mbox{\underline{\large$\triangleleft$}}}
\def\unrhd{\mbox{\underline{\large$\triangleright$}}}
\def\Join{\mbox{$\triangleright\triangleleft$}}
\def\mho{\mbox{\hspace*{-.04in} \rotatebox[origin=c] {180} {$\Omega$} }}

% Other special characters:
\def\cents{\mbox{{\large$c\!\!$}{\small /} }}
\def\half{{\mbox{\small{$\frac{1}{2}$}}}}
\def\third{{\mbox{\small{$\frac{1}{3}$}}}}
\def\fourth{{\mbox{\small{$\frac{1}{4}$}}}}
\def\quarter{{\mbox{\small{$\frac{1}{4}$}}}}
\def\fifth{{\mbox{\small{$\frac{1}{5}$}}}}
\def\sixth{{\mbox{\small{$\frac{1}{6}$}}}}
\def\thrhlf{{\mbox{\small{$\frac{3}{2}$}}}}
\def\twothrd{{\mbox{\small{$\frac{2}{3}$}}}}
\def\forthrd{{\mbox{\small{$\frac{4}{3}$}}}}
\def\fivthrd{{\mbox{\small{$\frac{5}{3}$}}}}
\def\thrfrth{{\mbox{\small{$\frac{3}{4}$}}}}
\def\twoffth{{\mbox{\small{$\frac{2}{5}$}}}}
\def\thrffth{{\mbox{\small{$\frac{3}{5}$}}}}
\def\forffth{{\mbox{\small{$\frac{4}{5}$}}}}

\def\chkbox{\mbox{$\Box\hspace{-.2em}^\surd\;$}}
\def\defeq{\mbox{$\stackrel{\rm def}{=}$}}
\def\hence{\mbox{\footnotesize{$_\bullet$$^\bullet$$_\bullet$ }}}
%\def\LR{{\rlap{\rm I}\hskip0.11em\hbox{\rm R}}}
%\def\LZ{{\mbox{\rm\bf Z}}}
\newcommand\LR{\ensuremath{\mathbb{R}}}
\newcommand\LQ{\ensuremath{\mathbb{Q}}}
\newcommand\LZ{\ensuremath{\mathbb{Z}}}
\def\subdot{\mbox{\tiny$\bullet$}}
\def\swiggle{\mbox{$\zeta$}}

% The following enforce consistent notation in math mode:
\def\dim{\mbox{{\rm dim}}}           % dimension
%\def\rank{\mbox{{\rm rank}}}         % rank
%\def\nul{\cN}                        % null space
%\def\col{\mbox{{\rm col}}}           % column space
%\def\row{\mbox{{\rm row}}}           % row space
%\def\diag{\mbox{{\rm diag}}}         % diagonal matrix
\def\goto{\rightarrow}               % goto (limit or map)
\def\imply{\mbox{$\,\Rightarrow\,$}} % imply
\def\iff{\mbox{$\Leftrightarrow$}}   % if, and only if
\def\compl{\mbox{$^{_\sim}\!$}}      % complement (also for www)
\def\evec{\mbox{\rm{e}}}             % e-vector
\def\min{\mbox{\rm{min}}}            % minimize
\def\max{\mbox{\rm{max}}}            % maximize
\def\opt{\mbox{\rm{opt}}}            % optimize
\def\inf{\mbox{\rm{inf}}}            % infimum
\def\sup{\mbox{\rm{sup}}}            % supremum
%\def\argmax{\mbox{\rm{argmax}}}      % argmax
%\def\argmin{\mbox{\rm{argmin}}}      % argmin
\def\det{\mbox{\rm{det}}}            % determinate
%\def\ri{\mbox{\rm{ri}}}              % relative interior
\def\affh{\mbox{\rm{affh}}}          % affine hull
\def\convh{\mbox{\rm{convh}}}        % convex hull
\def\Ext{\mbox{\rm{Ext}}}            % extreme points
\def\Binv{{\mbox{$B^{-1}$}}}         % basis inverse
\def\transp{\ensuremath{^{\mathsf{T}}}} % transpose
%%%%%%%%%%%%%%%%%%%%%%%%%%% End defs %%%%%%%%%%%%%%%%%%%%%%%%%%%%%%%

%%%%%%%%%%%%%%%%%%% Begin newcommand specs %%%%%%%%%%%%%%%%%%%%%%%%%

% Functions:
\newcommand{\cmd}[1]{\mbox{$\backslash${\sf{#1}}}}
\newcommand{\norm}[1]{{\left\|{#1}\right\|}}
\newcommand{\Bigfrac}[2]{\mbox{$\frac{\mbox{$#1$}}{\mbox{$#2$}}$}}
\newcommand{\card}[1]{\mbox{\large{\bf $|$}}\mbox{$\{#1\}$}\mbox{\large{\bf $|$} }}
\newcommand{\intval}[1]{\mbox{$\left\lfloor #1 \right\rfloor$}}
\newcommand{\intup}[1]{\mbox{$\left\lceil #1 \right\rceil$}}
\newcommand{\vx}[2]{\mbox{$\left(\begin{array}{c}
                                    \!\!#1\!\! \\ \!\!#2\!\! 
                                 \end{array}\right)$ }}
\newcommand{\bi}[1]{\textit{\textbf{#1}}}
\newcommand{\vect}[2]{\left(\!\begin{array}{rr} #1 \\ 
                                                #2 \end{array}\!\right)}
\newcommand{\mbf}[1]{\mbox{{\boldmath$#1$}}}
\newcommand{\brackets}[1]{\left\{ {#1} \right\}}
\newcommand{\set}[2]{\brackets{{#1} \left|\; {#2} \right.}}


% Graphics & fonts
\newcommand{\usc}[1]{\ul{\hspace*{#1 in}}}
\newcommand\hr[1]{\vspace{-.2in}\begin{center}
                        \begin{tabular}{c}\hspace*{#1in} \\ \hline
                        \end{tabular}\end{center} }
\newcommand{\smiley}{  $\;\;\;^{\begin{picture}(0,0)
     \put(-8,0){\circle{9}}
     \put(-9,1){\circle*{1}} \put(-6,1){\circle*{1}}
     \put(-11.5,-3.6){\tiny$\smile$}
                             \end{picture} }$} 
\newcommand{\sadface}{ $\;\;\;^{\begin{picture}(0,0)
     \put(-8,0){\circle{9}}
     \put(-9,1){\circle*{1}} \put(-6,1){\circle*{1}}
     \put(-11.5,-3.6){\tiny$\frown$}
                             \end{picture}}$}
% The following allows me to put a note in the margin.
\newcommand{\MarginNote}[1]{ \hskip 1sp 
           \marginpar{\footnotesize\sffamily\raggedright \textit{#1}} }
%%%%%%%%%%%%%%%%%%%%%%%%%%% End of file (eof) %%%%%%%%%%%%%%%%%%%%%%%%%%%%%%
